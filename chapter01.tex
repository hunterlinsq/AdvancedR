\documentclass[]{article}
\usepackage{lmodern}
\usepackage{amssymb,amsmath}
\usepackage{ifxetex,ifluatex}
\usepackage{fixltx2e} % provides \textsubscript
\ifnum 0\ifxetex 1\fi\ifluatex 1\fi=0 % if pdftex
  \usepackage[T1]{fontenc}
  \usepackage[utf8]{inputenc}
\else % if luatex or xelatex
  \ifxetex
    \usepackage{mathspec}
  \else
    \usepackage{fontspec}
  \fi
  \defaultfontfeatures{Ligatures=TeX,Scale=MatchLowercase}
\fi
% use upquote if available, for straight quotes in verbatim environments
\IfFileExists{upquote.sty}{\usepackage{upquote}}{}
% use microtype if available
\IfFileExists{microtype.sty}{%
\usepackage{microtype}
\UseMicrotypeSet[protrusion]{basicmath} % disable protrusion for tt fonts
}{}
\usepackage[margin=1in]{geometry}
\usepackage{hyperref}
\hypersetup{unicode=true,
            pdftitle={Advanced R 研讨},
            pdfauthor={罗智超Rokia.org},
            pdfborder={0 0 0},
            breaklinks=true}
\urlstyle{same}  % don't use monospace font for urls
\usepackage{graphicx,grffile}
\makeatletter
\def\maxwidth{\ifdim\Gin@nat@width>\linewidth\linewidth\else\Gin@nat@width\fi}
\def\maxheight{\ifdim\Gin@nat@height>\textheight\textheight\else\Gin@nat@height\fi}
\makeatother
% Scale images if necessary, so that they will not overflow the page
% margins by default, and it is still possible to overwrite the defaults
% using explicit options in \includegraphics[width, height, ...]{}
\setkeys{Gin}{width=\maxwidth,height=\maxheight,keepaspectratio}
\IfFileExists{parskip.sty}{%
\usepackage{parskip}
}{% else
\setlength{\parindent}{0pt}
\setlength{\parskip}{6pt plus 2pt minus 1pt}
}
\setlength{\emergencystretch}{3em}  % prevent overfull lines
\providecommand{\tightlist}{%
  \setlength{\itemsep}{0pt}\setlength{\parskip}{0pt}}
\setcounter{secnumdepth}{0}
% Redefines (sub)paragraphs to behave more like sections
\ifx\paragraph\undefined\else
\let\oldparagraph\paragraph
\renewcommand{\paragraph}[1]{\oldparagraph{#1}\mbox{}}
\fi
\ifx\subparagraph\undefined\else
\let\oldsubparagraph\subparagraph
\renewcommand{\subparagraph}[1]{\oldsubparagraph{#1}\mbox{}}
\fi

%%% Use protect on footnotes to avoid problems with footnotes in titles
\let\rmarkdownfootnote\footnote%
\def\footnote{\protect\rmarkdownfootnote}

%%% Change title format to be more compact
\usepackage{titling}

% Create subtitle command for use in maketitle
\newcommand{\subtitle}[1]{
  \posttitle{
    \begin{center}\large#1\end{center}
    }
}

\setlength{\droptitle}{-2em}
  \title{Advanced R 研讨}
  \pretitle{\vspace{\droptitle}\centering\huge}
  \posttitle{\par}
\subtitle{第一章 简介}
  \author{罗智超Rokia.org}
  \preauthor{\centering\large\emph}
  \postauthor{\par}
  \date{}
  \predate{}\postdate{}

\usepackage{xeCJK}

\begin{document}
\maketitle

{
\setcounter{tocdepth}{2}
\tableofcontents
}
\section{github地址}\label{github}

\url{https://github.com/zhichaoluo/AdvancedR/blob/master/chapter01.Rmd}

\section{前言}

感谢R语言社区,让我发现人世间居然有件兵器可以给自己终身修炼的动力,实乃人生幸事。AdvancedR就是我在练功过程中无意中发现的一本《易筋经》。爱不释手,字字珠玑。

市面上大部分有关R语言的书主要分为三类,一类是偏基础,一类偏概率统计(含数据挖掘、统计学习),另外一类是偏编程的。前面两类的汗牛充栋数量太大,就不推荐了。

\begin{itemize}
\item
  最后一类的首推Hadley Wickham写的这本《Advanced R》;
\item
  第二本是John M. Chamber的 《Software for Data Analysis》,R
  语言的前生S语言就是Chamber大大开发的,你说他的书能不看么?Chamber大大2016年新出的《ExtendingR》也是\textbf{强烈推荐},书中详细介绍了R语言的底层技术,包括函数编程、面向对象编程以及扩展方面的内容,其中专门有一章介绍了R语言的前世今生。其实Chamber大大还有两本书非常有名,1988年的《The
  New S Language》(江湖人称the blue book),1991年的《Statistical Methods
  in S》(江湖人称the white book);
\item
  第三本是Norman Matloff的《the Art of R
  programming》,Matloff教授拿到的是统计学博士却在计算机系教书,计算机功底杠杠的,看他的书,你会发现他不会放过任何一个角落来优化每一段代码。Matloff还擅长并行计算研究,他还出版了一本《Parrallel
  Computing for data science, with example in R, C++ and CUDA》
\item
  第四本是Patrick Burns的《R Inferno》,教你如何绕坑而不掉进坑里。
\item
  最权威R官方资料网站:\url{http://cran.r-project.org/doc/manuals/}
\end{itemize}

为什么只推荐偏编程的书呢?个人感觉,无论哪种软件都是你手中的一件兵器,无论哪种兵器,你只有熟练掌握它,它才能为你所用,学习R语言如果不将底层的编程弄明白,就像练功不懂内功心法,也就只有依葫芦画瓢,摆摆架势了。另外,底层编程熟练后,在此基础上将新研究的统计模型、方法应用在业务领域更是一马平川。(预告:下个研讨内容An
Introduction to Statistical Learning: with Application in R)

一直很反感充斥在市面上的一堆什么七天学会xx功夫的书籍,其实也不能怪这些书籍,练功者自己内心急躁,急功近利,自然会选择这样的书籍,有了需求,自然就有了供给。现在的小孩估计不看武侠小说,只打网游。我现在仍然清晰记得小李飞刀李寻欢\textbf{永不离手}的那把破旧的雕刻刀,虽然破旧平凡,但在百晓生兵器谱中排名第三。

``小李飞刀,例不虚发。刀光一闪,小李飞刀已发出,刀已插入他的咽喉,他瞪大眼睛,至死也不相信,没有人看清小李飞刀是如何出手的!''

古人云,天将降大任,必先冻心忍性。让我们用一学期的时间来好好研读《Advanced
R》,打造我们手中的``小李飞刀''。

\section{目标读者}

\begin{itemize}
\item
  想深入学习R并学习解决各种问题的新策略的中级R程序员
\item
  正在学习R,并想知道R为什么这样工作的其他语言的程序员
\end{itemize}

\section{可以学到什么?}

\begin{itemize}
\item
  熟悉R的基础
\item
  理解函数式编程
\item
  掌握元编程
\item
  性能优化
\item
  理解大多数R代码
\end{itemize}

\section{推荐阅读}

\begin{itemize}
\item
  《The Structure and Interpretation of Computer Programs》(SICP) by
  Harold Abelson, Gerald Jay Sussman
\item
  《Concepts, Techniques and Models of Computer Programming》 by Peter
  van Roy, Sef Haridi
\item
  《The Pragmatic Programmer》by Andrew Hunt, David Thomas
\end{itemize}

\section{获取帮助}

\begin{itemize}
\item
  \url{http://stackoverflow.com}
\item
  \url{http://github.com/hadley/adv-r}
\item
  研讨组地址:\url{http://github.com/zhichaoluo/AdvancedR}
\end{itemize}

\section{研讨组规则}

\begin{itemize}
\item
  三人行,必有我师
\item
  集体学习,集体讨论
\item
  欢迎提问,没有问题就有问题
\item
  必须参与练习,否则没有共鸣。
\item
  三次没有提交作业自动退出研讨组
\end{itemize}


\end{document}
